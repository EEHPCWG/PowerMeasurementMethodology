
\noindent
Refer to this section for detailed information about the elements of a power submission. 
\wl

\noindent
This section describes the information that must be included with a power measurement submission. It also describes some optional information that submitters may decide to include.
\wl

\noindent
The section contains definitions of the terms used to describe the elements of a power submission, some background information, motivation about why the list contains the elements it does, and any other details that may be helpful.
\noindent


\section{Measuring Device Specifications}
\noindent
Measuring devices must meet the Level requirements as defined in Sections 3.3 and 3.4. This section lists resources for finding and evaluating meters.
\wl

\noindent
The ANSI specification for revenue-grade meters is ANSI C12.20. 
\wl

\noindent
Also, refer to the Power and Temperature Measurement Setup Guide and the list of accepted power measurement devices from the Standard Performance Evaluation Corporation.
\wl

\noindent
\url{http://www.spec.org/power_ssj2008/}

\noindent
\url{http://www.spec.org/power/docs/SPECpower-Device_List.html }
\wl

\section{Measuring Device Terminology}
\noindent
Levels 1 and 2 specify power measurements. Level 3 specifies an energy measurement, but reports a power value.
\wl

\subsection{Sampling}
\noindent
For Levels 1 and 2, power measurements must be sampled at least once per second. The actual measurements that constitute a sample may be taken much more frequently than once per second. 
\wl

\noindent
Sampling in an AC context requires a measurement stage that determines the true power delivered at that point and enters that value into a buffer where it is then used to calculate average power over a longer time.  So "sampled once per second" in this context means that the times in the buffer are averaged and recorded once per second.
Sampling delivered electrical power in a DC context refers to a single simultaneous measurement of the voltage and the current to determine the delivered power at that point.  The sampling rate in this case is how often such a sample is taken and recorded internally within the device.  
\wl

\noindent
If the submitter is sampling in a DC context, most likely it will be necessary to adjust for power loss in the AC/DC conversion stage. Refer to Section 3.7 Aspect 3: Subsystems Included in Instrumented Power for details.
\wl

\newpage

\subsection{Power-Averaged and Total Energy Measurements}
\noindent
The reported power values for Levels 1 and 2 are power-averaged measurements. A power-averaged measurement is one taken by a device that samples the instantaneous power used by a system at some fine time resolution for a given interval. The power-averaged measurement for the interval is the numerical average of all the instantaneous power measurements during that interval and constitutes one reported measurement covering that interval. 
\wl

\noindent
Consider Level 1, which requires only one reported power value. This reported power value may consist of several power measurements taken at a frequency of at least once per second and averaged over an interval. That interval must be at least 20% of the run or one minute, whichever is longer. For example, if power measurements may be taken once per second for one minute for a total of 60 power measurements and then averaged, Level 1 reports one power-averaged value.
\wl

\noindent
Level 2 also requires that power measurements be taken at least once per second. Level 2 requires that power values be reported for both the core phase of the workload and the total workload. The reported power value for the core phase must be the result of at least 10 power measurements. These 10 power measurements may themselves be power-averaged measurements.
\wl

\noindent
Each of the required equally spaced measurements required for L2 must power-average over the entire separating space.  All the values reported by the meter must be used in the calculation.
\wl

\noindent
For example, the meter may sample at one-second intervals and report a value every minute. Assume that the core phase is 600 minutes long. Assume further that the requirement for 10 equally spaced measurements can be satisfied with 10 measurements spaced 50 minutes apart. Using just those 10 measurements does not conform to this specification because all the values reported by the meter during the core phase are not used. 
\wl

\noindent
Although those two measurements were equally spaced over 500 minutes , each averaged only over a minute. So some (the majority) of the separating space between the measurements was not included in the average. 
\wl

\noindent
For Levels 1 and 2, the units of the reported power values are watts.
\wl

\noindent
Level 3 specifies a total energy measurement that, when divided by the measured time, also reports power. An integrated measurement is a continuing sum of energy measurements. Typically, there are hundreds of measurements per second.  Depending on the local frequency standard, there must be at least 120 or 100 measurements per second. The measuring device samples voltage and current many times per second and integrate those samples to determine the next total energy consumed reading. 
\wl

\noindent
Level 3 reports an average power value for the core phase, an average power value for the whole run, at least 10 equally spaced energy values within the core phase, and the elapsed time between the initial and final energy readings in the core phase. The average power value for the core phase is the difference between the initial and final energy readings divided by the elapsed time.
\wl

\section{Aspect and Quality Levels}
\noindent
Table 3 1 summarizes the aspect and quality levels introduced in Section  1 Introduction
\wl

%INSERT TABLE 1
\noindent
\begin{table}
\caption{Summary of aspects and quality levels}
\label{tab:levels}
\begin{tabular}{|p{3.0cm}|p{3.5cm}|p{3.5cm}|p{3.5cm}|} \hline
\textbf{Aspect}&\textbf{Level 1}&\textbf{Level 2}&\textbf{Level 3}\\ \hline

\textbf{1a:~Granularity} &
One power sample per second &
One power sample per second &
Continuously integrated energy\\
\hline


\textbf{1b:~Timing} &
The longer of one minute or 20\% of the run &
Equally spaced across the full run &
Equally spaced across the full run   \\
\hline

\textbf{1c:~Measurements} &
Core phase average power &
$\bullet$ 10 average power measurements in the core phase &
$\bullet$  10 energy measurements in the core phase\\

 & &
$\bullet$  Full run average power &
$\bullet$  Full run average power \\

 & &
$\bullet$  idle power &
$\bullet$  idle power \\
\hline

\textbf{2: Machine \newline fraction}  &
The greater of 1/64 of the compute subsystem or 1 kW  &
The greater of 1/8 of the compute-node subsystem or 10 kW  &
The whole of all included subsystems \\
\hline

\textbf{3:~Subsystems} &
Compute-nodes only &
All participating subsystems, either measured or estimated &
All participating subsystem must be measured \\
\hline

\textbf{4: Point of measurement} &
Upstream of power conversion &
Upstream of power conversion &
Upstream of power conversion\\

 &
\centering \textbf{OR} &
\centering \textbf{OR} &
\centering \textbf{OR} \tabularnewline

 &
Conversion loss modeled with manufacturer data &
Conversion loss modeled with off-line measurements of a single power supply &
Conversion loss measured simultaneously \\
\hline
\end{tabular}
\end{table}

\section{Aspect 1: Granularity, Timespan and Reported Measurements}
\noindent
Aspect 1 has the following three parts. Levels 1, 2, and 3 satisfy this aspect in different ways.
\wl

\begin{itemize}
\item
The granularity of power measurements. This aspect determines the number of measurements per time element.
\item
The timespan of power measurements. This aspect determines where in the time of the workload's execution the power measurements are taken.
\item
The reported measurements. This aspect describes how the power measurements are reported.
\end{itemize}

\noindent
For all required measurements, the submission must also include the data used to calculate them. For Level 2 and Level 3 submissions, the supporting data must include at least 10 equally spaced points in the core of the run. 
\wl

\noindent
Levels 2 and 3 require a number of equally spaced measurements to be reported for two reasons.  
\wl

\begin{itemize}
\item
One is that facility or infrastructure level power measurements are typically taken by a system separate from the system OS and thus cannot be easily synchronized with running the benchmark.  
\item
Secondly, with multiple periodic measurements, more reporting points are included before and after the benchmark run to ensure that a uniform standard of "beginning" and "end" of the power measurement can be applied to all the power measurements on a list.  
\end{itemize}

\noindent
There is no maximum number of reported points, although one reported measurement per second is probably a reasonable upper limit. The submitter may choose to include more than 10 such points.
\wl

\noindent
The number of reported average power measurements or total energy measurements is deliberately given large latitude.  Different computational machines will run long or short benchmark runs, depending on the size of the machine, the memory footprint per node, as well as other factors.  Typically the power measurement infrastructure is not directly tied to the computational system's OS and has its own baseline configuration (say, one averaged measurement every five minutes).  These requirements are specified not only to give a rich data set but also to be compatible with typical data center power measurement infrastructure.
\wl

\noindent
All levels specify that power measurements be performed within the core phase of a workload. Levels 2 and 3 specify that a power measurement for the entire application be reported. Consequently, these levels require measurements during the run but outside of the core phase.
\wl

\subsection{Core Phase}
\noindent
All submissions must include the average power within the core (parallel computation) phase of the run. 
\wl

\noindent
Every workload has a core phase where it is maximally exercising the relevant component(s) of the system. More power is often consumed in a workload's core phase than in its startup and shutdown phases. The core phase typically coincides with maximum system power draw. 
\wl

\noindent
For example, the core phase of the HPL workload is the portion of the core that actually solves the matrix. It is the numerically intensive solver phase of the calculation. Note that HPL now contains an HPL\_timer() routine that facilitates power measurements.
\wl
