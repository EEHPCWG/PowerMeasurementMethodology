\chapter{Reporting Power Values \normalsize{(Detailed Information)}}
\label{sec:reporting}

This section contains:
\begin{itemize}
 \item The exact requirements for each quality level of each aspect.
 \item A description of what must be included with a power measurement submission.
       It also describes some optional information that submitters may decide to include.
 \item Definitions of the terms used to describe the elements of a power submission, some background information, motivation about why the list contains the elements it does, and any other details that may be helpful.
\end{itemize}

\section{Measuring Device Terminology and Specifications}
\label{sec:MDSpecs}
\label{sec:MDTerm}

This section defines meter accuracy requirements.
First, Subsection~\ref{sec:accuracy} defines the general accuracy requirements.
The following section describe the different types of measurements and the required sampling rates. 
Measuring devices must also meet the Level requirements as defined in Sections~\ref{sec:AQLevels} and \ref{sec:A1GTRM}.

\subsection{Accuracy Requirements}
\label{sec:accuracy}

For Level~3, it is required to use any revenue grade meter, any meter accepted by SPEC power, or any meter documented to have an accuracy of \SpecAccuracyLThree{} or better.
Level~2 requires a meter with a minimum documented accuracy of \SpecAccuracyLTwo{}.
Level~1 requires a meter with a minimum documented accuracy of \SpecAccuracyLOne{}.

The required accuracy as percentage refers to the relative error specified for the power meter.
If the power meter has multiple measurement intervals with different accuracies, all measurement intervals used during the benchmark run must meet the above requirements.
Error in this context refers to the sum of statistical and systematic error.
It is acceptable if the error is documented as a hard bound, as the 1$\sigma$ standard deviation, or as the full width at half maximum (FWHM).

All requirements and operating conditions that are required for the power meter to achieve its documented accuracy must be met.
For Levels~1 and~2, it is acceptable to assume that the AC net power is a sine wave function.

If the site uses multiple independent power meters for a Level~2 or a Level~3 measurement, and if all employed power meters measure an identical fraction of the system, the requirements are relaxed in the following way.
\begin{packed_item}
\item 
Each individual power meter must have at least a documented accuracy of \SpecAccuracyMeter{}.
\item
The documented error of the individual meters divided by the square root of the number of employed meters must meet the above requirements of \SpecAccuracyLTwo{} for Level~2 and \SpecAccuracyLThree{} for Level~3.
\end{packed_item}
In this case, the second requirement ensures that the standard deviation of the measurement of the entire run, using error propagation, and assuming a Gaussian distribution, and assuming that the errors of the power meters are uncorrelated, meets the requirements.
For example, assume 4 power meters for a Level~3 measurement, each measuring \SI{25}{\percent} of the total system.
In that case, each meter must have a documented accuracy of~$\SI{1}{\percent} \cdot \sqrt{4} = \SI{2}{\percent}$ to get a final accuracy of~\SI{1}{\percent}.
In other words, you can use a single meter with \SI{1}{\percent} accuracy, \num{4} meters with \SI{2}{\percent} accuracy, or \num{9} meters with \SI{3}{\percent} accuracy to reach~\SI{1}{\percent} for Level~3.

Revenue grade meters are defined by ANSI C12.20.
For meters accepted by the SPEC power, refer to the {\itshape Power and Temperature Measurement Setup Guide \/} and the list of accepted power measurement devices from the Standard Performance Evaluation Corporation.
\begin{packed_item}
\item 
\url{http://www.spec.org/power_ssj2008/}
\item
\url{http://www.spec.org/power/docs/SPECpower-Device_List.html }
\end{packed_item}

\subsection{Power Readings and Sampling}
\label{sec:PAaTEM}

The timing aspect of the power quality levels mandate at what intervals a power value is read from the measurement device and processed in software, the \emph{readout interval}.
Each of the processed power values may be calculated by the device internally from multiple samples.
This document does not cover the specific internal sampling rates.
Instead, the accuracy aspect of power quality levels implies that internal sampling rates are appropriate to the given operating conditions, especially AC or DC power supply.
The readout interval should be approximately constant for the duration of the observation.

A power value read from a measurement device can represent the power at a specific moment (\emph{instantaneous power}) or an \emph{average power} for a certain interval (e.g. the last \SI{1}{\second}) or a \emph{running average} over a longer time window.
However, it is often not clearly documented by devices, which is the case.
Therefore, there is no strict requirement for Levels~1 and~2.

For Level~3, the submitter should have a good understanding of the nature of the measured power values to ensure that the computed final average power over the core phase is unaffected by errors from aliasing or other problems with the timing of read values.
Consider for example a device which measures \SI{50}{\hertz} AC-Power for 10 cycles (\SI{0.2}{\second}) should also be read every \SI{0.2}{\second} to ensure an accurate average power.

It is also allowed to read values more often than they are updated.
For example, reading a power value every \SI{1}{\second}, but ten consecutive values are identical, because the device only updates it every \SI{10}{\second}.
However, this only qualifies as \SI{10}{\second} intervals and therefore Level~1.

\subsubsection{Reporting Average Power}

The primary reported power values used for the efficiency calculation are average power values over the core phase or the full run (\emph{observation interval}).
The value is typically computed from the numerical average of the aforementioned series of power readings (which may itself be average power values computed by the device for shorter time intervals).

For computing the average power value exactly those values that lie within the core phase / full run time interval should be used.
Some advanced corner cases may warrant some additional consideration as outlined in the following.
The overall goal is to compute an average power value that is as close as possible to the true physical value --- as in the energy consumed during the benchmark divided by the duration.
Values at the boundaries of the observation interval \emph{may} be weighted by the overlap of the value's individual averaging time and the observation interval, if known.
If the precision of the benchmark timestamps is less than the measurement, e.g., only integral seconds rather than fractions, you \emph{may} correct for rounding, e.g., by adding \SI{500}{\milli\second} to the timestamps.

While the main ranking is computed from the overall average power value, submitters are encouraged to include the series of individual power values and their timestamps in the submission, e.g., as a csv file.

\subsubsection{Using Continuously Integrated Energy}

\emph{Continuously integrated energy} is computed on the measurement device itself.
The internals of how the energy is computed should be part of the accuracy specification of the measuring device and depends also on the instrumentation location.
It should be ensured that any assumptions made by a measuring device regarding the accuracy, e.g. sinusoidal input voltage for AC or frequency ranges, are met.

Continuously integrated energy may be used at all power quality levels.
Unlike with power values, the readout rate is not as significant.
However, energy values must be read within ten seconds (Level~1) or one second (Level~2 and~3) of the beginning and end of the core phase, respectively, to ensure overall accuracy.

Using integrated energy values is encouraged, when available.
The reported value is still the average power, which is computed by dividing the difference in energy measurement by the difference in timestamps of the last/first reading within the observation interval.
Note that the duration of the measurement may differ slightly from the duration of the benchmark as measurement timestamps do not align perfectly with beginning and end of the benchmark.
In that case, make sure to use the timestamps from measurements rather than the benchmark.

\begin{equation}
	P_{\text{avg}} = \frac{E_\text{measure}}{t_\text{measure}} = \frac{E_\text{end} - E_\text{begin}}{t_\text{end} - t_\text{begin}}
\end{equation}

Submitters are encouraged to include the series of intermediate energy values during the core phase and their timestamps in the submission, e.g., as a csv file.

\section{Aspect and Quality Levels}
\label{sec:AQLevels}

Table~\ref{tab:levels} summarizes the aspect and quality levels introduced in Section~\ref{sec:intro}. 


\newlength{\tablecolumnwidth}
\setlength{\tablecolumnwidth}{3.5cm}
\noindent
\begin{table}
\caption{Summary of aspects and quality levels}
\label{tab:levels}
\begin{tabular}{|p{2.5cm}|p{3.5cm}|p{3.5cm}|p{3.5cm}|} \hline
\textbf{Aspect}&\textbf{Level~1}&\textbf{Level~2}&\textbf{Level~3}\\ \hline

\textbf{1a: Timing} &
\begin{minipage}[t]{\tablecolumnwidth}
      \begin{compactitemize}
            \item every 60~s
            \item[] \textbf{and}
            \item $\ge10$~measurements during core phase
      \end{compactitemize}
\end{minipage} &
\multicolumn{2}{p{2\tablecolumnwidth}|}{
      \begin{minipage}[t]{2\tablecolumnwidth}
            \begin{compactitemize}
                  \item power every \SI{1}{\second}
                  \item[] \textbf{or}
                  \item continuously integrated energy read within \SI{1}{\second} of begin/end
            \end{compactitemize}
      \end{minipage}
} \\
\hline


\textbf{1b: Reporting} &
\begin{minipage}[t]{\tablecolumnwidth}
      Average power
      \begin{compactitemize}
            \item core phase
      \end{compactitemize}
\end{minipage} &
\multicolumn{2}{p{2\tablecolumnwidth}|}{
      \begin{minipage}[t]{2\tablecolumnwidth}
            Average power
            \begin{compactitemize}
                  \item core phase
                  \item ful run
                  \item idle
            \end{compactitemize}
      \end{minipage}
} \\
\hline


% \textbf{1a:~Granularity} &
% One power sample per second &
% One power sample per second &
% Continuously integrated energy\\
% \hline


% \textbf{1b:~Timing} &
% At equal intervals across the entire core phase of the run, which must be at least one minute &
% At equal intervals across the full run &
% At equal intervals across the full run   \\
% \hline

% \textbf{1c:~Measurements} &
% Core phase average power &
% $\bullet$ Core phase average power & 
% $\bullet$ Core phase total energy \\ %MDSpecs

%  & &
% $\bullet$ Full run average power &
% $\bullet$ Full run total energy \\

%  & &
% $\bullet$ \MinMeasurementsCorePhaseLTwoThree{} average power measurements in the core phase &
% $\bullet$ \MinMeasurementsCorePhaseLTwoThree{} total energy measurements in the core phase \\

%  & &
% $\bullet$ Idle power &
% $\bullet$ Idle power \\
% \hline

\textbf{2: Machine \newline fraction}  &
The entire system; at least \SpecPowerMaxLOne{}; or the largest of \SpecFracMinLOne{} of the compute subsystem, \SpecPowerMinLOne{}, or \SpecMinNodes{}~compute nodes &
The greater of \SpecFracMinLTwo{} of the compute-node subsystem or \SpecPowerMinLTwo{} or \SpecMinNodes{}~compute nodes - alternatively the entire system &
The whole of all included subsystems \\
\hline

\textbf{3:~Subsystems} &
Compute-nodes measured, interconnect measured or estimated &
Compute-nodes measured, all participating subsystems measured or estimated &
All participating subsystem must be measured \\
\hline

\textbf{4: Location of measurement} &
Upstream of power conversion &
Upstream of power conversion &
Upstream of power conversion\\

 &
\centering \textbf{OR} &
\centering \textbf{OR} &
\centering \textbf{OR} \tabularnewline

 &
Conversion loss modeled with manufacturer data &
Conversion loss modeled with off-line measurements of a single power supply &
Conversion loss measured simultaneously \\
\hline
\textbf{4b: Meter accuracy} &
\SpecAccuracyLOne{} &
\SpecAccuracyLTwo{} (see Section~\ref{sec:MDSpecs})&
\SpecAccuracyLThree{} (see Section~\ref{sec:MDSpecs})\\
\hline
\end{tabular}
\end{table}

\section{Aspect~1: Granularity, Timespan and Reported Measurements}
\label{sec:A1GTRM}

Aspect~1 has the following three parts.
Levels~1, 2, and~3 satisfy this aspect in different ways.

\begin{itemize}
\item
The granularity of power measurements.
This aspect determines the number of measurements per time element.
\item
The timespan of power measurements.
This aspect determines where in the time of the workload's execution the power measurements are taken.
\item
The reported measurements.
This aspect describes how the power measurements are reported.
\end{itemize}

For all required measurements, the submission must also include the data used to calculate them.
For Level~2 and Level~3 submissions, the supporting data must include at least \MinMeasurementsCorePhaseLTwoThree{} measurement values at a regular sampling rate in the core phase of the run.

Levels~2 and~3 require reported measurement values at regular intervals for the following reasons:

\begin{itemize}
\item
Facility or infrastructure level power measurements are typically taken by a system separate from the system OS and thus cannot be easily synchronized with running the benchmark.
\item
With multiple periodic measurements, more measurement values are included before and after the benchmark run.
This ensures that a uniform standard of \textit{beginning} and \textit{end} of the power measurement can be applied to all the power measurements from different sensors.
\end{itemize}

There is no maximum number of reported measurement values, although one reported measurement per second is a reasonable upper limit.
The submitter may choose to include more than \MinMeasurementsCorePhaseLTwoThree{} such measurement values.

The number of reported average power measurements or total energy measurements is deliberately given large latitude.
Different computational machines will run long or short benchmark runs, depending on the size of the machine and the memory footprint per node, as well as other factors.
Typically the power measurement infrastructure is not directly tied to the computational system's OS and has its own baseline configuration (say, one averaged measurement every five minutes).
These requirements are specified not only to give a rich data set but also to be compatible with typical data center power measurement infrastructure.

All levels specify that power measurements be performed within the core phase of a workload.
Levels~2 and~3 specify that a power measurement for the entire application be reported.
Consequently, these levels require measurements during the run but outside of the core phase.

\subsection{Core Phase}
\label{sec:core_phase}

All submissions must include the average power within the core phase of the run.

The core phase is usually considered to be the section of the workload that undergoes parallel execution.
It typically does not include the parallel job launch and teardown and is required to run for at least one minute.

In order to correlate the power measurement and the performance measurement to obtain the power efficiency, power and performance must be measured over the exact same time period.
Therefore, the core phase is defined as the time period that is used for the performance calculation.
Sometimes, the exact same time is impossible to measure due to power meter restrictions.
Therefore, Level~1 and Level~2 allow a small part at the beginning and at the end which is not measured (see Section~\ref{sec:PAaTEM}).

For example, the core phase of the Linpack workload is the portion of the code that actually solves the matrix.
It is the numerically intensive solver phase of the calculation.
In case of the standard HPL implementation of the Linpack benchmark as provided by netlib.org (\url{http://www.netlib.org/benchmark/hpl/}) this is the time spent in the routing {\ttfamily HPL\_pdgesv}.
Note that HPL as of version 2.1 contains a timer routine that prints start and end time of the core phase to facilitate power measurements, which looks like:\newline
{\ttfamily
HPL\_pdgesv() start time Wed Jun 30 21:49:08 2015\newline
HPL\_pdgesv() end time Wed Jun 30 22:23:15 2015
}.

\subsection{The Whole Run}

Level~2 and Level~3 submissions must also include the average power for the whole run, from the initiation of the job to its completion.

Since HPL only reports the time spent executing the core phase of the workload, the time for the total run must be measured and reported separately by the submitter, for example by prepending the UNIX time command to the job invocation or the parallel application launch, whichever is earlier.

Levels~2 and~3 require the entire run to be measured and reported because HPL (the default workload at the time this document was written) drops significantly in power consumption during the course of a computation, as the matrix size being computed gets smaller.
Requiring the entire run eliminates systematic bias caused by using different parts of the run for the measurement.
Figure~\ref{fig:powprof} shows an example of a power graph taken from an HPL run on the LLNL Muir system in the fall of 2011.

Note that the power drops by \SI{8}{\percent} or so during the computational phase of the run.
This graph also illustrates the need for the device measurement to have as high a time resolution as possible.
The spread of power measurements even within a small time span is very likely caused by the sampling going in and out of phase with the AC input power.
Figure~\ref{fig:sopm} shows a smaller time slice that clearly illustrates this.


%INCLUDE FIG 3-1
\begin{figure}
\centering
\includegraphics[width=4in]{fig3-1}
\caption{Power Profile HPL Run}
\label{fig:powprof}
\end{figure}

%INCLUDE FIG 3-2
\begin{figure}
\centering
\includegraphics[width=4in]{fig3-2}
\caption{Spread of Power Measurements}
\label{fig:sopm}
\end{figure}

The boxes are individual one-second power samples.
This fast up-down fluctuation is not caused by the behavior of an individual power supply; the power being sampled is over the entire computer system doing the HPL run.
The reason that Level~3 requires integrating total energy meters is because they measure at high enough sampling rates to not be subject to these sampling artifacts.

Level~3 requires a power-meter device with a higher inherent measurement granularity as compared to Levels~1 and~2.
Levels~1 and~2 allow 1-second intervals of the input power.
Level~3 requires an integrating total-energy meter, which samples the input power multiple times per AC cycle and so is much less susceptible to sampling artifacts caused by the AC waveform.

\subsection{Level~1}

Level~1 submissions include a measurement of the average power during the core phase.
The device measurement granularity must be at least one instantaneous measurement of power per second.
This requirement holds whether the measurement is DC or AC.

There must be at least one power-averaged measurement during the run.
The total interval covered must be the core phase of the run which must be at least one minute long.
Exclude power readings for time intervals that do not lie completely in the core phase.

\subsection{Level~2}

Level~2 submissions include a measurement of the average power during the core phase of the run, the average power during the full run, and a series of regular power-averaged measurements.
The device measurement granularity must be at least one instantaneous measurement of power per second.
This requirement holds whether the measurement is DC or AC.

The submitted value for the core phase must be the average of all power readings taken during the core phase of the run.
In addition, the average of all power readings during the full run must be submitted.
In both cases, exclude power readings for time intervals that do not lie completely in the respective time period of core phase and full run.
The core phase is required to run for at least one minute.

On top of these full measurements, a series of power-averaged measurements of equal length at regular intervals must be submitted for the full run.
These intervals must be short enough so that at least \MinMeasurementsCorePhaseLTwoThree{} measurements are reported during the core phase of the workload.
This series of measurements in total must cover the full run.

Figure~\ref{fig:a1l2pm} illustrates the series of power measurements at regular intervals of Aspect~1 of a Level~2 power measurement.
Each measurement is an average of instantaneous power measurements, and these instantaneous measurements are taken once per second.
As an example, the figure shows \num{10} power-averaged measurements within the core phase and four power-averaged measurements outside the core phase, two before the core phase and two after the core phase.

%INCLUDE FIG 3-4
\begin{figure}
\centering
\includegraphics[width=4in]{fig3-4}
\caption{Aspect~1 of Level~2 Power Measurements}
\label{fig:a1l2pm}
\end{figure}

\subsection{Level~3}

Level~3 submissions include a measurement of the average power during the core phase of the run and the average power during the full run.

The complete set of total energy readings (at least \MinMeasurementsCorePhaseLTwoThree{} during the core computation phase) must be included, along with the execution time for the core phase and full run.

Level~3 requires continuously integrated total energy measurements rather than power-averaged measurements.
The readings must begin before the start of the run and extend to when it is finished.

The intervals at which total energy readings are reported must be short enough so that at least \MinMeasurementsCorePhaseLTwoThree{} reported readings fall within the core phase of the workload.
Note that each reported energy reading is the integral over many internal power samples.

The measured energy for the core phase is the last measured total energy within the core phase minus the first measured total energy within the core phase.
The final power is calculated by dividing this energy by the elapsed time between these first and last energy readings.
These last and first measurements in the core phase must be timed such that no more than a total of ten seconds (five each at the beginning and end) of the core phase are not covered by the total energy measurement.
The procedure for the full run is analogous.

Figure~\ref{fig:a1l3pm} illustrates Aspect~1 of  Level~3 power measurement.
The figure shows \num{10} readings in the core phase of the workload.
Note that these are integrated readings.
To obtain a power reading, one must subtract two integrated readings and divide by the time between the readings.

%INCLUDE FIG 3-5
\begin{figure}[t]
\centering
\includegraphics[width=4in]{fig3-5}
\caption{Aspect~1 of Level~3 Power Measurements}
\label{fig:a1l3pm}
\end{figure}

\section{Format of Reported Measurements}
\label{sec:FoRM}

Levels~2 and~3 require the complete set of measurements.
The submitter may choose to provide these values in a CSV file.
Do not provide scans of paper documents.

The submitter may find it useful to create a graph showing the power and energy during the workload as shown in Figure~\ref{fig:powengwl}.
Keep this graph for reference, but do not provide it as part of the submission.


%INCLUDE FIG 3-6
\begin{figure}
\centering
\includegraphics[width=4in]{fig3-6}
\caption{Power and Energy During the Workload}
(used with permission from Universit\'{e} Laval, Calcul Qu\'{e}bec, Compute Canada)
\label{fig:powengwl}
\end{figure}

\section{Aspect~2: Machine Fraction Instrumented}
\label{sec:A2MFI}

Aspect~2 specifies the fraction of the system whose power feeds are instrumented by the measuring equipment.

Level~3 requires that the entire machine be measured.
Level~2 requires a higher fraction than Level~1.

Levels~1 and~2 do not measure the power of the entire system, but require only a measurement of a fraction of the system and estimate the total power consumption by extrapolation.
When calculating the average power of the full machine for Levels~1 and~2, the measured power must be divided by this fraction to extrapolate to the average power drawn by the whole machine.
For example, if the submitter measures the power delivered to $\frac{1}{4} $ of the machine, the submitter must then multiply the measured power by \num{4} to obtain the power for the whole machine.
The higher machine fractions are required at the higher quality levels to reduce the effects of random fluctuations and minor differences in hardware influencing the power measurements.
The larger the sample, the more transients will tend to cancel out.

The requirements with respect to the compute-node subsystem for each quality level are as follows.

\begin{itemize}
\item
Level~1: Any of
\begin{itemize}
\item The entire machine
\item At least \SpecPowerMaxLOne{}
\item Whichever is largest of: a minimum of \SpecPowerMinLOne{} of power, \SpecFracMinLOne{} of the system, or \SpecMinNodes{} nodes
\end{itemize}

\item
Level~2: The largest of \SpecFracMinLTwo{} of the compute-node subsystem or at least \SpecPowerMinLTwo{} or \SpecMinNodes{} compute nodes.
Alternatively, the entire system if smaller than \num{15} nodes.
\item
Level~3: The power use of the whole machine must be measured.
\end{itemize}

For measurements of other subsystems (network, storage, etc.), measure at least \SpecFracMinLOne{} for Level~1, \SpecFracMinLTwo{} for Level~2, and everything for Level~3.

The fraction of a subsystem that is measured must be chosen such, that the entire subsystem is a real multiple of the fraction.
For instance, a cluster consisting of blade servers with \num{8} nodes per chassis requires a measurement of the entire chassis, hence a multiple of \num{8} nodes.
Even if individual node power could be measured, that would not be valid as it would not include the common chassis.

The requirements for Level~1 and Level~2 include the measurement of at least \SpecMinNodes{} nodes.
For a cluster consisting of individual servers, a node is simply defined as one server.
There are architectures with multiple individual servers in one chassis, e.g.
2U twin or twin$^2$ servers with \num{2} and \num{4} servers per chassis and blade servers with many more nodes in one chassis.
Usually, all the nodes in one chassis share redundant power supplies.
Hence, in that case you have to measure the entire chassis, including all the power supplies of a chassis.
The measurement of a chassis is then considered a measurement of as many nodes as there are nodes in the chassis.

For a measurement of only a fraction of a subsystem, this fraction must be chosen randomly.
In this context, random refers to the entity that is measured.
For instance, if you measure nodes then choose random nodes, if you measure entire racks, choose random racks.
It is not acceptable to screen the system for components with lower power consumption and then measure only that fraction, nor is it acceptable to replace components in the chosen fraction with others that have lower power consumption.

\section{Aspect~3: Subsystems Included in Instrumented Power}
\label{sec:A3SIiIP}

Aspect~3 specifies the subsystems included in the instrumented power.

Subsystems in the context of this document are power subsystems.
A power subsystem is that part of a supercomputer which can be measured in isolation for power consumption while the supercomputer is performing a task.

Subsystems include computational nodes, any interconnect network the application uses, any head or control nodes, any storage system the application uses, and any internal cooling devices (self-contained liquid cooling systems and fans).

If some subsystems are part of the measured power, their power may not be subtracted out after the measurement.
The explicitly measured value must be used as is.

For Level~1, both the compute-node subsystem and the interconnect must be reported.
The compute-node subsystem power must be measured.
The interconnect subsystem participating in the workload must also be measured or, if not measured, the contribution must be estimated.
Include everything that you need to operate the interconnect network that is not part of the compute subsystem.

For Level~2, measure the compute node subsystem.
Other subsystems participating in the workload must be measured or estimated.

For Level~3, all power going to the parts of a computer system that participate in a workload must be included in the power measurement.
The reported power measurement must include all computational nodes, any interconnect network the application uses, any head or control nodes, any storage system the application uses, all power conversion losses inside the computer, and any internal cooling devices (self-contained liquid cooling systems and fans).

For Levels~1 and~2, estimations of other subsystems besides the compute subsystem must be performed by substituting the measurement by an upper bound derived from the maximum specified power consumption of all hardware components.

Measurements may include infrastructure that is shared.
It is allowed but not required to exclude subsystems that don't participate in the workload (e.g. storage subsystems).
However, if these subsystems are part of the cabinet or rack being measured, they may not be excluded even if they are not used.
That is, the submitter cannot calculate their contribution and subtract that contribution.
If the subsystem is not part of the rack or cabinet being measured and it does not participate in the workload, it need not be measured.
For some systems, it may be impossible not to include a power contribution from certain subsystems that are not used by the application.
In this case, provide a list of the measured subsystems, but do not subtract an estimated value for the subsystems that are not used.

For example, the node board may include compute nodes and GPUs, and the application may not actually use the GPUs.
If you cannot easily shut down the GPUs (say with an API), you must still include the power that they use.
It is not acceptable to measure the power for both the compute nodes and the GPUs and then subtract the GPU power from the measurement.

A site may include more subsystems than are strictly required if it chooses or if it is advantageous from a measurement logistics point of view.

In other words, for Level~2 and~3 you have to include everything that you cannot turn of during the benchmark run.
It only allowed to exclude subsystems (or parts thereof) from the measurement in the first place, but their contribution must not be subtracted afterwards.
For Level~1 this applies as well, but only for the network subsystem.

\subsection{Heterogeneous Systems}

A particular system may have different types of compute nodes.
The system may have compute nodes from different companies or even compute nodes with different architectures.
These compute nodes are said to belong to different heterogeneous sets.
Each heterogeneous set must consist of identical nodes.

With Level~3, the submitter need not be concerned about heterogeneous sets of compute nodes because Level~3 measures the entire system.

Levels~1 and~2, however, measure a portion of the compute-node subsystem and estimate contributions from unmeasured portions.
With Levels~1 and~2, the submitter must measure at least one member of each heterogeneous set.
Measure at least \SpecFracMinLOne{} of each heterogeneous set for Level~1, and at least \SpecFracMinLTwo{} of each heterogeneous set for Level~2.
Then, extrapolate the total power per each heterogeneous set as you would do for a non heterogeneous cluster.
Finally, accumulate the contributions of all heterogeneous sets.

For example, assume there exist two sets of compute nodes, a set called A and another called B.
The submitter is able to measure the power 
consumed by $\frac{1}{2} $ of the A compute nodes and $\frac{1}{4}$ of the B compute nodes.

The total power measurement reported for compute nodes would then be 

\[ Total~power=2*(power~from~compute~nodes~A) + 4*(power~from~compute~nodes~B) \]

The assumption of Levels~1 and~2 is that all the compute nodes in a set react identically to the workload.


\section{Aspect~4: Instrumentation Location where the Electrical Measurements are Taken}
\label{sec:A4wEMaT}
Aspect~4 specifies where in the power distribution system the power delivery is measured.
For all quality levels, the submission indicates where power is measured and the quantity of parallel measurements.

Measurements of power or energy are typically made at multiple locations in parallel across the computer system.
For example, such locations can be at the entrance to each separate rack, or at the output connector of multiple building transformers.

All the reported measurements taken in parallel at a given instant in time are then summed into a total measurement for that time.
The total measurement for a given moment in time constitutes one entry in the series of measurements that becomes part of the submission.

AC measurements are upstream of the system's power conversion.
If the measurements are in a DC context, the submitter may have to take into account some power loss.
Refer to Section~\numnameref{sec:AfPL}.

Electrical power or energy measurements shall be taken in one of the following locations.

\begin{itemize}
\item[{A)}]
At a location upstream of where the electrical supply from the data center is delivered to the computer system

OR

\item[{B)}]
At the location of the electrical supply delivery to the computer system from the data center

OR

\item[{C)}]
At a location just inside of the computer system that is electrically equivalent to location B) above.
This includes the following.

\begin{itemize}
\item
At any location within a passive PDU, at the input to the PDU, at the output connector(s) of the PDU, or anywhere in between 
\item
At the input connector to the first power-modifying component (for example, the Blue Gene bulk power module, Cray AC/DC converters, and possibly the input connector to one or more crate power supplies)
\end{itemize}
\end{itemize}

If the measuring device or devices used to satisfy Aspect~1 also meet the ABC location requirements specified above, then those devices are sufficient to obtain the measurements needed for submission.

Refer to Section~\ref{sec:MDSpecs} for the specifications of the required accuracy of the power meter devices.

\subsection{Adjusting for Power Loss}
\label{sec:AfPL}

If the measurement device(s) that satisfy Aspect~1 are downstream of the ABC locations specified above, then two sets of measurements must be taken in order to determine the power loss between the required and the actual measurement location.

\begin{itemize}
\item
For a Level~1 measurement, the power loss may be a load-varying model based on specifications from the manufacturer.
\item
For a Level~2 measurement, the power loss may be a load-varying model based on an actual physical measurement of one power supply in the system.
\item
For a Level~3 submission, the power loss must be measured simultaneously by a device at the required location (one of the ABC locations) measured at least once every five minutes, averaged long enough to average out the AC transients.
\end{itemize}

For all three levels, the power losses used and how they were determined must be part of the submissions.

\subsection{Data Center Schematic}

Figure~\ref{fig:powmeasschem} is an example of a simple power measurement schematic.
This example shows only one power measurement location.

Submitters may find it useful to create such a schematic to identify the power measurement locations.
Keep this schematic for reference, but do not provide it as part of the submission.

%INCLUDE FIG 3-7
\begin{figure}
\centering
\includegraphics[width=4in]{fig3-7}
\caption{Example of a Power Measurement Schematic}
(used with permission from Universit\'{e} Laval, Calcul Qu\'{e}bec, Compute Canada)
\label{fig:powmeasschem}
\end{figure}


\section{Environmental Factors}
\label{sec:EF}

Reporting information about the cooling system temperature is optional.
It is requested to provide a description of the cooling system as well as where and how the temperature was measured.
Reporting temperature allows for better comparison of reported results, as there is a indirect correlation between temperature and power.

All other environmental data is optional.
Other environmental data may include factors such as:

\begin{packed_item}
\item[{-}]
\% deviation between supply and rated voltage and frequency (recommended $\pm$\,5\%)
\item[{-}]
\% total harmonic distortion (recommended \textless\,2\% THD )
\item[{-}]
line impedance (recommended \textless\,0.25 ohm)
\item[{-}]
relative humidity
\end{packed_item}

