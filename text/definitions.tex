\chapter{Definitions}
\label{sec:definitions}

\setlength{\parskip}{1pt}
\paragraph*{Core Phase}
\begin{quote}
The core phase is defined as the time period that is used for the performance calculation in the benchmark.
\end{quote}

\paragraph*{CSV file}
\begin{quote}
A comma-separated values (CSV) file stores tabular data (numbers and text) in plain-text form.
A CSV file is readable by most spreadsheet programs.
\end{quote}

\paragraph*{Metric}
\begin{quote}
A basis for comparison; a reference point against which other things can be evaluated; a measure.
\end{quote}

\paragraph*{Methodology}
\begin{quote}
The system of methods followed in a particular discipline; a way of doing something, especially a systematic way; implies an orderly logical arrangement (usually in steps); a measurement procedure.
\end{quote}


\paragraph*{Network Time Protocol (NTP)}
\begin{quote}
Network Time Protocol (NTP) is a networking protocol for clock synchronization between computers.
\end{quote}

\paragraph*{Power-Averaged Measurement}
\begin{quote}
A power-averaged measurement over a certain time is the numerical average of all power readings of a power meter reported during that time.
The power meter internally samples the instantaneous power used by a system at some fine time resolution (say, once per second).
The power meter either reports every such power sample as a power reading or averages all power samples taken during a certain interval and then reports only one single power reading as measurement for that interval.
\end{quote}

\paragraph*{Sampling and Power Readings}
\begin{quote}
Sampling delivered electrical power in a DC context refers to a single simultaneous measurement of the voltage and the current to determine the delivered power at that point.
The sampling rate in this case is how often such a sample is taken and recorded internally within the device.
Sampling in an AC context requires a measurement stage, whether analog or digital, that determines the true power delivered at that point and enters that value into a buffer where it is then used to calculate average power over a longer time.
So ``sampled once per second'' in this context means that the times in the buffer are averaged and recorded once per second.

The power meter may either report every power sample that is measured.
In that case, power sample and power reading are the same.
Or the power meter may internally average all power samples over a certain interval and then report only one power reading covering that entire interval.
\end{quote}

\paragraph*{Workload}
\begin{quote}
The application or benchmark software designed to exercise the HPC system or subsystem to
the fullest capability possible.
\end{quote}

