\chapter{Introduction}
\label{sec:intro}

\noindent
This document recommends a methodological approach for measuring, recording, and reporting the power used by a high performance 
computer (HPC) system while
running a workload. 
\wl

\noindent
This document is part of a collaborative effort between the Green500, the Top500, 
the Green Grid, and the Energy Efficient High Performance Computing Working Group.  While 
it is intended for this methodology to be generally applicable to benchmarking a variety of 
workloads, the initial focus is on High Performance LINPACK (HPL). 
\wl

\noindent
This document defines four aspects of a power measurement and three quality ratings. All 
four aspects have increasingly stringent requirements for higher quality levels.  
\wl

\noindent
The four aspects are as follows:


\begin{packed_enum}
\item 
Granularity, time span, and type of raw measurement
\item 
Machine fraction instrumented
\item 
Subsystems included in instrumented power
\item 
Where in the power distribution network the measurements are taken
\end{packed_enum}

The quality ratings are as follows:

\begin{packed_item}
\item 
Adequate, called Level~1 (L1)
\item
Moderate, called Level~2 (L2)
\item
Best, called Level~3 (L3)
\end{packed_item}

\fbox{
\begin{minipage}{0.75\linewidth}
The requirements for all four aspects for a given quality
level must be satisfied to grant that quality rating for a submission.
\end{minipage}
}
