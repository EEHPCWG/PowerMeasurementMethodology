\chapter{Introduction}
\label{sec:intro}

This document recommends a methodological approach for measuring, recording, and reporting the power used by a high performance computer (HPC) system while running a workload.

This document is part of a collaborative effort between the Green500, the Top500, the Green Grid, and the Energy Efficient High Performance Computing Working Group (EEHPC WG).
While it is intended for this methodology to be generally applicable to benchmarking a variety of workloads, the initial focus is on High Performance LINPACK (HPL).

This document defines four aspects of a power measurement and three quality levels.
All four aspects have requirements that become increasingly stringent at higher quality levels. 

The four aspects are as follows:

\begin{packed_enum}
\item 
Granularity, time span, and type of raw measurement
\item 
Machine fraction instrumented
\item 
Subsystems included in instrumented power
\item 
Location of measurement in the power distribution network and accuracy of power meters.
\end{packed_enum}

The quality ratings are as follows:

\begin{packed_item}
\item 
Adequate, called Level~1 (L1)
\item
Moderate, called Level~2 (L2)
\item
Best, called Level~3 (L3)
\end{packed_item}

\textbf{To grant a given quality level for a submission, the submission must
satisfy the requirements of all four aspects at that quality level
or higher.}
