\chapter{Introduction}
\label{sec:intro}

This document recommends a methodological approach for measuring, recording, and reporting the power used by a \ac{HPC} system while running a workload.

This document is part of a collaborative effort between the Green500, the Top500, the Green Grid, and the Energy Efficient High Performance Computing Working Group (EEHPC WG).
While it is intended for this methodology to be generally applicable to benchmarking a variety of workloads, the initial focus is on \ac{HPL}.

Benchmarking computer systems is a way of comparing performance between architectures.
There are benchmarks that are used for comparing the energy efficiency of architectures and include both a measure of the benchmark performance and either energy or power.
Benchmarking efforts, like the Top500 and the Green500, provide a ranked list of general purpose systems that are in common use for high end applications.
They also provide a reliable basis for tracking and detecting trends in high performance computing.

Besides providing more accurate and precise information for the Green500 List, there are other reasons to make a system-level power measurement.
There are a lot of capabilities today for measuring component-level power.
This is especially true for processors and accelerators, but it also includes node and rack-level measurements.
System-level power is still a challenge for many sites and systems.
Sites and vendors may want to do architectural trending and system modeling for design, selection, upgrades, tuning and analysis.
Another example is the use for understanding the actual --- versus peak --- power draw of an HPC system based on a normal site workload vs. wall-plate power.
This might be useful for provisioning the datacenter as there can be a substantial difference between peak and normal power draw. Operation improvements, such as for cooling systems, can leverage the ability to measure HPC system-level power and energy.
Finally, system-level measurements can be used to validate component-level measurements by summing them up.


\section{Content of this document}


This document defines four aspects of a power measurement and three quality levels.
All four aspects have requirements that become increasingly stringent at higher quality levels. 

The four aspects are as follows:

\begin{packed_enum}
\item 
Granularity, time span, and type of raw measurement
\item 
Machine fraction instrumented
\item 
Subsystems included in instrumented power
\item 
Location of measurement in the power distribution network and accuracy of power meters.
\end{packed_enum}


The quality ratings are as follows:

\begin{packed_item}
\item 
Adequate, called Level~1
\item
Moderate, called Level~2
\item
Best, called Level~3
\end{packed_item}

\textbf{To grant a given quality level for a submission, the submission must satisfy the requirements of all four aspects at that quality level or higher.}
