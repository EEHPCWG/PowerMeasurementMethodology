
\noindent
When you are ready to make a submission, go to the Top500 
(\url{http://www.top500.org/}) 
and Green500 (\url{http://www.green500.org/}) sites for more information. 
\wl

\noindent
This section contains a checklist for all the items of information you need 
to consider when making a power measurement. 
\wl

\noindent
Read through the list and ensure that you can record the needed 
information when you run your workload.
\wl

\begin{itemize}
\item[{[ ]}]
\textbf{Quality Level}
\addcontentsline{toc}{subsection}{Quality Level}

Choosing a quality level is the first important decision a submitter must make. Refer to 
Section~\ref{sec:AQLevels} Aspect and Quality Levels for general information about the three quality levels.  
Sections~\ref{sec:A1GTRM} through \ref{sec:A4wEMaT} describe the details of the three quality levels

\item[{[ ]}]
\textbf{Power Measurement Locations}
\addcontentsline{toc}{subsection}{Power Measurement Locations}

Measurements of power or energy are often made at multiple points in parallel across the computer system. A typical location might be the output of the building transformer.
Refer to Section~\ref{sec:A4wEMaT} Aspect 4: Point where the Electrical Measurements are Taken 
for more information about power measurement locations.

Note that in some cases, you may have to adjust for power loss. For information about power loss, refer to 
Section~\ref{sec:AfPL} Adjusting for Power Loss.  If you adjust for power loss, how you determined the power losses must be part of the submission.


\newpage

\item[{[ ]}]
\textbf{Measuring Devices}
\addcontentsline{toc}{subsection}{Measuring Devices}

Specify the measuring device or devices used. A reference to the device specifications is useful.

Refer to Section~\ref{sec:MDTerm} for some terminology about the measuring device specific to the power submissions described in this document. This section describes the difference between power-averaged measurements and total energy measurements.

Refer to Section~\ref{sec:MDSpecs} for information about the required measuring device.

If multiple meters are used, describe how the data aggregation and synchronization were performed. One possibility is to have the nodes NTP-synchronized; the power meter's controller is then also NTP-synchronized prior to the run.

\item[{[ ]}]
\textbf{Workload Requirement}
\addcontentsline{toc}{subsection}{Workload Requirement}

The workload must run on all compute nodes of the system. Level 3 measures the power for the entire system. Levels 1 and 2 measure the power for a portion of the system and extrapolate a value for the entire system. 

\item[{[ ]}]
\textbf{Level 3 Power Measurement}
\addcontentsline{toc}{subsection}{Level 3 Power Measurement}

Level 3 submissions include the average power during the core phase of the run and the average power during the full run.

The core phase is usually considered to be the section of the workload that undergoes parallel execution. The core phase typically does not include the parallel job launch and teardown.

Level 3 measures energy. Power is calculated by dividing the measured energy by the elapsed time . The measured energy is the last measured total energy within the core phase minus the first measured total energy within the core phase.

Refer to Section~\ref{sec:PAaTEM} Power-Averaged and Total Energy Measurements for information about the distinction between energy and power.

The complete set of total energy readings used to calculate average power (at least 10 during the core computation phase) must be included, along with the execution time for the core phase and the execution time for the full run.

Refer to Section~\ref{sec:A1GTRM} Aspect 1: Granularity, Timespan and Reported Measurement for more 
information about the Level 3 Power Submission.

Refer to Section~\ref{sec:FoRM} for more information about the format of reported measurements.

For Level 3, all subsystems participating in the workload must be measured. Refer to 
Section~\ref{sec:A3SIiIP} Aspect 3: Subsystems Included in Instrumented Power for more information about included subsystems.

With Level 3, the submitter need not be concerned about different types of compute nodes because Level 3 measures the entire system.

\newpage
\item[{[ ]}]
\textbf{Level 2 Power Measurement}
\addcontentsline{toc}{subsection}{Level 2 Power Measurement}

Level 2 submissions include the average power during the core phase of the run and the average power during the full run.

The complete set of power-averaged measurements used to calculate average power must also be provided. 
Refer to Section~\ref{sec:PAaTEM} Power-Averaged and Total Energy Measurements for the definition of a power-averaged measurement and how it differs from a total energy measurement.

For Level 2, the workload run must have a series of equally spaced power-averaged measurements of equal length. These power-averaged measurements must be spaced close enough so that at least 10 measurements are reported during the core phase of the workload. The reported average power for the core phase of the run is the numerical average of the 10 (or more) power-averaged measurements collected during the core phase.

Each of the required equally spaced measurements required for L2 must power-average over the entire separating space. 

Refer to Section~\ref{sec:A1GTRM} Aspect 1: Granularity, Timespan and Reported Measurement for more information about the Level 2 Power Submission. 

Refer to Section~\ref{sec:FoRM} for more information about the format of reported measurements.

For Level 2, all subsystems participating in the workload must be measured or estimated. Level 2 requires that the 
greater of $ \frac{1}{8} $ of the compute-node subsystem or 10 kW of power be measured. It is acceptable to exceed this requirement.

The compute-node subsystem is the set of compute nodes. As with Level 1, if the compute-node subsystem contains different types of compute nodes, you must measure at least one member from each of the heterogeneous sets. The contribution from compute nodes not measured must be estimated. Refer to Section~\ref{sec:A3SIiIP} Aspect 3: Subsystems Included in Instrumented Power for information about heterogeneous sets of compute nodes.


\newpage
\item[{[ ]}]
\textbf{Level 1 Power Measurement}
\addcontentsline{toc}{subsection}{Level 1 Power Measurement}

Level 1 requires at least one power-averaged measurement during the run. Refer to 
Section~\ref{sec:PAaTEM} Power-Averaged and Total Energy Measurements for the definition of a power-averaged measurement and how it differs from a total energy measurement.

The total interval covered must be at least 20\% of the core phase of the run or one minute, whichever is {\itshape longer\/}. 

If the choice is one minute (because it's longer than 20\% of the core phase) that minute must reside in the middle 80\% of the core phase. If the middle 80\% of the core phase is less than one minute, the measurement must include the entire middle 80\% and overlap equally on both sides.

If the choice is 20\% of the core phase (because this 20\% is greater than one minute), this 20\% must reside in the middle 80\% of the core phase.

Refer to Section~\ref{sec:A1GTRM} Aspect 1: Granularity, Timespan and Reported Measurement for more information about the Level 1 power submission.

For Level 1, the only subsystem that requires an actual power measurement is the compute-node subsystem. 
The compute-node subsystem is the set of compute nodes. Measure the greater of $ \frac{1}{64} $ of the compute-node subsystem or 1kW of power.

 
The interconnect subsystem participating in the workload must be measured or, if not measured, the contribution 
must be estimated. The interconnect is everything that you need to operate the network that is not part of the compute subsystem. 
This may include infrastructure that is shared, but excludes the part that is not servicing the current cluster.

List all other subsystems that contribute to the workload, but you do not have to provide 
measured or estimated values for their contribution.

For some systems, it may be impossible not to include a power contribution from some subsystems. In this case, list what you are including, but do not subtract an estimated value for the included subsystem.

If the compute node-subsystem contains different types of compute nodes, measure at least one member from each of the heterogeneous sets. The contribution from compute nodes not measured must be estimated. Refer to 
Section~\ref{sec:A3SIiIP} Aspect 3: Subsystems Included in Instrumented Power for information about
 heterogeneous sets of compute nodes.


\item[{[ ]}]
\textbf{Idle Power}
\addcontentsline{toc}{subsection}{Idle Power}

Idle power is defined as the power used by the system when it is not running a workload, but it is in a state where it is ready to accept a workload. The idle state is not a sleep or a hibernation state.

An idle measurement need not be linked to a particular workload. The idle measurement need not be made just before or after the workload is run. Think of the idle power measurement as a constant of the system. Think of idle power as a baseline power consumption when no workload is running. 

For Levels 2 and 3, there must be at least one idle measurement. An idle measurement is optional for Level 1. 

\newpage
\item[{[ ]}]
\textbf{Included Subsystems}
\addcontentsline{toc}{subsection}{Included Subsystems}

Subsystems include (but are not limited to) computational nodes, any interconnect network the application uses, any head or control nodes, any storage system the application uses, and any internal cooling devices (self-contained liquid cooling systems and fans).  

\begin{itemize}
\item
For Level 1, all subsystems participating in the workload must be listed. 

Only the compute-node subsystem must be measured. Not every compute node belonging to the compute node subsystem must be measured, but the contribution from those compute nodes not measured must be estimated.  Measure the greater of at 
least $ \frac{1}{64} $ of the compute-node system or at least 1kW of power. 

\item
For Level 2, all subsystems participating in the workload must be measured and, if not measured, their contribution must be estimated. The Measured \% and the Derived \% must sum to the Total \%.

In the case of estimated measurements for subsystems other than 
the compute-node subsystem, the submission must include the relevant 
manufacturer specifications and formulas used for power estimation. 

Measure the greater of at least 10KW of power or $ \frac{1}{3} $ of the compute-node subsystem.

\item
For Level 3, all subsystems participating in the workload must be measured.
\end{itemize}

Include additional subsystems if needed.

Refer to Section~\ref{sec:A3SIiIP} Aspect 3: Subsystems Included in Instrumented Power for more information about included subsystems.

Refer to Section~\ref{sec:A2MFI} Aspect 2: Machine Fraction Instrumented for information about \% requirements for Levels 1 and 2.
 
\item[{[ ]}]
\textbf{Tunable Parameters}
\addcontentsline{toc}{subsection}{Tunable Parameters}

Listing tunable parameters for all levels is optional.
Typical tunable values are the CPU frequency, memory settings, and internal network settings. Be conservative, but list any other values you consider important.

A tunable parameter is one that has a default value that you can easily change before running the workload.

If you report tunable parameters, submit both the default value (the value that the data center normally supplies) and the value to which it has been changed.

\newpage
\item[{[ ]}]
\textbf{Environmental Factors}
\addcontentsline{toc}{subsection}{Environmental Factors}

All levels require information about the cooling system temperature. Reporting other environmental data (such as humidity) is optional. 

Submissions require both the in and the out temperature of the cooling system. For air-cooled systems, these are the in and out air temperatures. For liquid-cooled systems, these are the in and out temperatures of the liquid.

Refer to Section~\ref{sec:EF} Environmental Factors for more information.

\end{itemize}
